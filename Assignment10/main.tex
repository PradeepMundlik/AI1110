%%%%%%%%%%%%%%%%%%%%%%%%%%%%%%%%%%%%%%%%%%%%%%%%%%%%%%%%%%%%%%%
%
% Welcome to Overleaf --- just edit your LaTeX on the left,
% and we'll compile it for you on the right. If you open the
% 'Share' menu, you can invite other users to edit at the same
% time. See www.overleaf.com/learn for more info. Enjoy!
%
%%%%%%%%%%%%%%%%%%%%%%%%%%%%%%%%%%%%%%%%%%%%%%%%%%%%%%%%%%%%%%%


% Inbuilt themes in beamer
\documentclass{beamer}

% Theme choice:
\usetheme{CambridgeUS}

% Title page details: 
\title{Assignment 10} 
\author{Pradeep Mundlik (AI21BTECH11022)}
\date{\today}
% \logo{\large \LaTeX{}}


\begin{document}

% Title page frame
\begin{frame}
    \titlepage 
\end{frame}

% Remove logo from the next slides
% \logo{}


% Outline frame
\begin{frame}{Outline}
    \tableofcontents
\end{frame}


% Lists frame
\section{Question}
\begin{frame}{Question}
    \begin{block}{Papoulis 8.10}
        Among 4000 newborns, 2080 are male. Find the 0.99 confidence interval of the probability p = P\{male\}.
    \end{block}
\end{frame}

\section{Solution}
\begin{frame}{Solution}
    The joint density
    \begin{align}
        f(X,c) = c^n e^{-cn\left(\bar{x} - x_0 \right)} \dots x_i > x_0
    \end{align}
    has an interior maximum if 
    \begin{align}
        &\frac{df(X,c)}{dc} = 0 \\
        \implies &\hat{c} = \frac{1}{\bar{x} - x_0} 
    \end{align}
\end{frame}

\begin{frame}
    Now, we have 
    \begin{align}
        &\bar{x} = 2080/4000 = 0.52 \\
        &n = 4000 \\
        &z_u \backsimeq 2.326 \\
        &P_{1,2} \backsimeq \bar{x} \pm z_u \sqrt{\frac{\bar{x}\left(1 - \bar{x}\right)}{n}} = 0.52 \pm 0.018 \\
        \implies &0.502 < p < 0.538 
    \end{align}
\end{frame}
\end{document}