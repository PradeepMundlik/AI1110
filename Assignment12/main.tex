%%%%%%%%%%%%%%%%%%%%%%%%%%%%%%%%%%%%%%%%%%%%%%%%%%%%%%%%%%%%%%%
%
% Welcome to Overleaf --- just edit your LaTeX on the left,
% and we'll compile it for you on the right. If you open the
% 'Share' menu, you can invite other users to edit at the same
% time. See www.overleaf.com/learn for more info. Enjoy!
%
%%%%%%%%%%%%%%%%%%%%%%%%%%%%%%%%%%%%%%%%%%%%%%%%%%%%%%%%%%%%%%%


% Inbuilt themes in beamer
\documentclass{beamer}

% Theme choice:
\usetheme{CambridgeUS}

% Title page details: 
\title{Assignment 12} 
\author{Pradeep Mundlik (AI21BTECH11022)}
\date{\today}
% \logo{\large \LaTeX{}}


\begin{document}

% Title page frame
\begin{frame}
    \titlepage 
\end{frame}

% Remove logo from the next slides
% \logo{}


% Outline frame
\begin{frame}{Outline}
    \tableofcontents
\end{frame}


% Lists frame
\section{Question}
\begin{frame}{Question}
    \begin{block}{Papoulis 9.36}
        Using (9.135), show that
        \begin{align}
            R(0) - R(\tau) \geq \frac{1}{4^n} [R(0) - R(2^n \tau)]
        \end{align}
        9.135
        \begin{align}
            S(w) = \int_{-\infty}^{\infty} R(\tau)\cos(\omega \tau) \,d\tau \\
            R(\tau) = \frac{1}{2\pi}\int_{-\infty}^{\infty} S(w)\cos(\omega \tau) \,d\omega
        \end{align}
    \end{block} 
\end{frame}

\section{Solution}
\begin{frame}{Solution}
        We will use : 
        \begin{align}
            1 - \cos(\theta) = 2\sin^2(\frac{\theta}{2}) \geq 2\sin^2(\frac{\theta}{2})\cos^2(\frac{\theta}{2}) = \frac{1}{4} (1 - cos(2\theta))
        \end{align}
        we will conclude with 9.135 that
        \begin{align}
            R(0) - R(\tau) &= \frac{1}{2\pi}\int_{-\infty}^{\infty} S(w)(1 - \cos(\omega \tau)) \,d\omega  \\
            &\geq \frac{1}{8\pi}\int_{-\infty}^{\infty} S(w)(1 - \cos(\omega \tau)) \,d\omega \\
            &= \frac{1}{4} [R(0) - R(2\tau)] \\
            R(0) - R(\tau) &\geq \frac{1}{4} [R(0) - R(2\tau)]
        \end{align}
\end{frame}
\begin{frame}
    the result follows n = 1. 
    Reapeating the above, we obtain general result
    \begin{align}
        R(0) - R(\tau) \geq \frac{1}{4^n} [R(0) - R(2^n \tau)]
    \end{align}
\end{frame}
\end{document}