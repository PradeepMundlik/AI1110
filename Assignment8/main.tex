%%%%%%%%%%%%%%%%%%%%%%%%%%%%%%%%%%%%%%%%%%%%%%%%%%%%%%%%%%%%%%%
%
% Welcome to Overleaf --- just edit your LaTeX on the left,
% and we'll compile it for you on the right. If you open the
% 'Share' menu, you can invite other users to edit at the same
% time. See www.overleaf.com/learn for more info. Enjoy!
%
%%%%%%%%%%%%%%%%%%%%%%%%%%%%%%%%%%%%%%%%%%%%%%%%%%%%%%%%%%%%%%%


% Inbuilt themes in beamer
\documentclass{beamer}

% Theme choice:
\usetheme{CambridgeUS}

% Title page details: 
\title{Assignment 8} 
\author{Pradeep Mundlik (AI21BTECH11022)}
\date{\today}
% \logo{\large \LaTeX{}}


\begin{document}

% Title page frame
\begin{frame}
    \titlepage 
\end{frame}

% Remove logo from the next slides
% \logo{}


% Outline frame
\begin{frame}{Outline}
    \tableofcontents
\end{frame}


% Lists frame
\section{Question}
\begin{frame}{Question}
    \begin{block}{Papoulis 5.44}
        The random variable x has zero mean, central moments $\mu_n$, and cumulants $\lambda_n$. Show that $\lambda_3 = \mu_3$, $\lambda_4$ = $\mu_4 - 3{\mu_2}^2$.
    \end{block}
\end{frame}

\section{Solution}
\begin{frame}{Solution}
    If $\eta = 0$, then $m_n = \mu_n$ \\
    \begin{align}
        &\lambda_1 = \eta = 0 \\
        &\phi(s) = \sum_{n = 0}^{\infty}  \frac{\mu_n}{n!} s^n \\
        &\psi(s) = \sum_{n = 2}^{\infty} \frac{\lambda_n}{n!} s^n \\
        &1 + \frac{\mu_2}{2!}s^2 + \frac{\mu_3}{3!}s^3 + \dots = exp\left(\frac{\lambda_2}{2!} s^2 + \frac{\lambda_3}{3!}s^3 + \dots\right)
    \end{align}
\end{frame}
\begin{frame}{Solution}
    Expanding the exponential and equating powers of s, we obtain
    \begin{align}
        &\mu_2 = \lambda_2 \\
        &\mu_3 = \lambda_3 \\
        &\frac{\mu_4}{4!} = \frac{\lambda_4}{4!} + \frac{1}{2!}\left(\frac{\lambda_2}{2!}\right)^2 \\
        \implies &\lambda_4 = \mu_4 - 3{\lambda_2}^2 \\
        \implies &\lambda_4 = \mu_4 - 3{\mu_2}^2 
    \end{align}
\end{frame}


\end{document}